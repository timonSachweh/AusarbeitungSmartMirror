%% ========================
% Globale Einstellungen. Müssen angepasst werden!
%% ========================
\newcommand{\erstgutachter}{Name des Erstgutachters}
\newcommand{\zweitgutachter}{Name des Zweitgutachters}
\newcommand{\autorname}{Name des/der Studenten/in}
% Umbruch im Titel mit "\\"
\newcommand{\titel}{Titel der Arbeit}
\newcommand{\arbeitstyp}{Bachelorarbeit}
% Die nächste Zeile einkommentieren, falls die Nennung einer kooperierenden Entität gewünscht ist.
%\renewcommand{\kooperation}{In Kooperation mit:\\Firma/Fakultät/Lehrstuhl}

%% ========================
%  Einstellungen für Verzeichnisse.
%  Hier können die gewünschten Verzeichnisse im Anhang an- und abgestellt werden.
%% ========================
% Abbildungsverzeichnis einfügen?
\setboolean{AbbildungsverzeichnisEinfuegen}{true}
% Tabellenverzeichnis einfügen?
\setboolean{AlgorithmenverzeichnisEinfuegen}{false}
% Algorithmenverzeichnis einfügen?
\setboolean{TabellenverzeichnisEinfuegen}{false}
% Verzeichnisse im Anhang statt immer zweiseitig auch einseitig zulassen, um leere Seite zu vermeiden? Wenn die Verzeichnisse sehr wenig enthalten, sollte auf "true" geschaltet werden.
\setboolean{PlatzsparendeVerzeichnisse}{true}

%% ========================
%  Sollen Kopfzeilen innen oder außen dargestellt werden? 
%  Wenn der Drucker keinen Druck bis an den Seitenrand erlaubt 
%  (gilt für die meisten Heimanwender-Drucker), sollte "true" gewählt werden,
%  sonst (gilt i.d.R. für das Drucken im Copy-Shop) sollte "false" gewählt werden.
%% ========================
\setboolean{KopfInnen}{false}

%% ========================
%  Sollen aufwändige (farblich unterlegte) Kapitelüberschriften verwendet werden?
%% ========================
\setboolean{UnterlegteKapitel}{false}

%% ========================
%  Sollen für Kopfzeilen und Kapitelüberschriften TU-Farben verwendet werden?
%  (Gilt nur für aufwändige Kapitelüberschriften.)
%% ========================
\setboolean{TuFarben}{false}
