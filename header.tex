% header.tex
\documentclass[a4paper,12pt,ngerman]{article}
% Dimensionen des Dokuments. Nicht ändern!
%\usepackage[a4paper,left=3.5cm,right=2.5cm,bottom=2.5cm,top=3.5cm,headheight=20pt]{geometry}

\usepackage[pdftex]{graphicx}

% Einige wichtige Standard-Pakete
\usepackage{amsmath,amssymb,subfigure}

% Theorem-Umgebungen
\usepackage[amsmath,thmmarks]{ntheorem}

% Korrekte Darstellung der Umlaute
\usepackage[utf8x]{inputenc}
\usepackage[ngerman]{babel}
\usepackage[T1]{fontenc}

% Minimale Verbesserung von Zeilenumbrüchen und Wortabständen
\usepackage{microtype}

% Formatierung von floats
\usepackage{floatrow}

% Algorithmen
\usepackage[plain,chapter]{algorithm}
\usepackage{algorithmic}

\usepackage{enumerate}
% Eine verbesserte Version der Schriftart computer modern; für Vektorschrift überall.
\usepackage{lmodern}

% Bibtex deutsch
\usepackage{bibgerm}

% Für Testzwecke
\usepackage{blindtext}

% urls
\usepackage[hidelinks, colorlinks=true, urlcolor=blue]{hyperref}

% Umfließender Text neben Bild
\usepackage{wrapfig}

% Multirow für Tabellen
\usepackage{multirow}

% Eurosymbol
\usepackage{eurosym}

% Zeilenabstand
\usepackage{setspace}

% Caption Packet
\usepackage[margin=0pt,font=small,labelfont=bf]{caption}

\usepackage[explicit]{titlesec}

%Für das Einbinden der Eidestattlichen Erklärung
\usepackage{pdfpages}

% Das fancy header paket für schönere Kopf- und Fußzeilen
\usepackage{fancyhdr}
\usepackage{helvet}
\usepackage{xcolor}
\usepackage{calc}

% ==============
% Es folgt die Konfiguration der Kopf- und Fußzeilen
% ==============
\definecolor{TUGreen}{rgb}{0.517,0.721,0.094}
\definecolor{bgpagenum}{gray}{.75}
\definecolor{fgheader}{gray}{.6}
\newboolean{TuFarben}
\setboolean{TuFarben}{false}


\def\mystrut(#1){\vrule height #1pt width 0pt}    
\newlength{\headerspace}
\setlength{\headerspace}{1mm}

\newboolean{KopfInnen}
\setboolean{KopfInnen}{false}

\newcommand{\textmarkleft}{\fontfamily{phv}\selectfont\fontsize{12}{5}\bfseries
	\colorbox{bgpagenum}{
		\begin{minipage}[t][8pt]{4cm}
		\raggedleft\textcolor{white}{\thepage}
		\end{minipage}
	}
	\hspace{\headerspace}
	\textcolor{fgheader}{
		\nouppercase\rightmark
	}
}
	
\newcommand{\textmarkright}{
	\fontfamily{phv}\selectfont\fontsize{12}{5}\bfseries
	\textcolor{fgheader}{
		\nouppercase\leftmark
	}
	\hspace{\headerspace}
	\colorbox{bgpagenum}{
		\begin{minipage}[t][8pt]{4cm}
			\raggedright\textcolor{white}{\thepage}	
		\end{minipage}
	}
}

\newcommand{\numbermarkright}{
	\fontfamily{phv}\selectfont\fontsize{12}{5}\bfseries
	\colorbox{bgpagenum}{
		\begin{minipage}[t][8pt]{4cm}
			\raggedright\textcolor{white}{\thepage}
		\end{minipage}
	}
}

\newcommand{\numbermarkleft}{
	\fontfamily{phv}\selectfont\fontsize{12}{5}\bfseries
	\colorbox{bgpagenum}{
		\begin{minipage}[t][8pt]{4cm}
			\raggedleft\textcolor{white}{\thepage}	
		\end{minipage}
	}
}

\newcommand{\fancypages}{
	\pagestyle{fancy}
	\ifthenelse{\boolean{KopfInnen}}{
		\fancyheadoffset[loh]{4.5cm}
		\fancyheadoffset[reh]{4.5cm}
	}{
		\fancyheadoffset[leh]{4.5cm}
		\fancyheadoffset[roh]{4.5cm}
	}
	\renewcommand{\headrulewidth}{0pt}
	\renewcommand{\footrulewidth}{0pt}
	\fancyhead[LO]{\ifthenelse{\boolean{KopfInnen}}{\textmarkleft}{}}
	\fancyhead[RO]{\ifthenelse{\boolean{KopfInnen}}{}{\textmarkright}}
	\fancyhead[RE]{\ifthenelse{\boolean{KopfInnen}}{\textmarkright}{}}
	\fancyhead[LE]{\ifthenelse{\boolean{KopfInnen}}{}{\textmarkleft}}
	\fancyfoot[C]{}
}

\newcommand{\configurefancypages}{
	\fancypagestyle{plain}{%
		\fancyhf{} 
		\ifthenelse{\boolean{KopfInnen}}{
			\fancyheadoffset[loh]{4.5cm}
			\fancyheadoffset[reh]{4.5cm}
		}{
			\fancyheadoffset[leh]{4.5cm}
			\fancyheadoffset[roh]{4.5cm}
		}
		\renewcommand{\headrulewidth}{0pt}
		\renewcommand{\footrulewidth}{0pt}
		
		\fancyhead[LO]{\ifthenelse{\boolean{KopfInnen}}{\numbermarkleft}{}}
		\fancyhead[RO]{\ifthenelse{\boolean{KopfInnen}}{}{\numbermarkright}}
		\fancyhead[RE]{\ifthenelse{\boolean{KopfInnen}}{\numbermarkright}{}}
		\fancyhead[LE]{\ifthenelse{\boolean{KopfInnen}}{}{\numbermarkleft} }
		\fancyfoot[C]{}
	}
	\ifthenelse{\boolean{TuFarben}}{
		\colorlet{bgpagenum}{TUGreen!70!white}
		\colorlet{fgheader}{TUGreen!80!white}
	}{}
}
% ==============
% Konfiguration der Kopf- und Fußzeilen abgeschlossen.
% ==============


% Konfiguration der Chapter-Überschriften 
% für den Fall, dass ausgefallene Überschriften gewünscht sind
\newboolean{UnterlegteKapitel}
\setboolean{UnterlegteKapitel}{false}
\newlength{\almosttextwidth}
\setlength{\almosttextwidth}{\textwidth-6pt}
\newcommand{\configurechapterheadings}{
\ifthenelse{\boolean{UnterlegteKapitel}}{
	
	\titleformat{\chapter}
	{\vspace{-0.75cm}\normalfont\Huge\bfseries}{}{0em}{\colorbox{bgpagenum}{\begin{minipage}[b][60pt]{\almosttextwidth}\strut\textcolor{white}{\thechapter\quad##1}\end{minipage}}}
	\titleformat{name=\chapter,numberless}
	{\vspace{-0.75cm}\normalfont\Huge\bfseries}{}{0em}{\colorbox{bgpagenum}{\begin{minipage}[b][60pt]{\almosttextwidth}\strut\textcolor{white}{##1}\end{minipage}}}
}{}
}

% Einige Einstellungen für Floats. Bitte nicht ändern!
\floatsetup[table]{capposition=top}
\floatsetup[algorithm]{capposition=top}

% Theorem-Optionen %
\theoremseparator{.}
\theoremstyle{change}
\newtheorem{theorem}{Theorem}[section]
\newtheorem{satz}[theorem]{Satz}
\newtheorem{lemma}[theorem]{Lemma}
\newtheorem{korollar}[theorem]{Korollar}
\newtheorem{proposition}[theorem]{Proposition}
% Ohne Numerierung
\theoremstyle{nonumberplain}
\renewtheorem{theorem*}{Theorem}
\renewtheorem{satz*}{Satz}
\renewtheorem{lemma*}{Lemma}
\renewtheorem{korollar*}{Korollar}
\renewtheorem{proposition*}{Proposition}
% Definitionen mit \upshape
\theorembodyfont{\upshape}
\theoremstyle{change}
\newtheorem{definition}[theorem]{Definition}
\theoremstyle{nonumberplain}
\renewtheorem{definition*}{Definition}
% Kursive Schrift
\theoremheaderfont{\itshape}
\newtheorem{notation}{Notation}
\newtheorem{konvention}{Konvention}
\newtheorem{bezeichnung}{Bezeichnung}
\theoremsymbol{\ensuremath{\Box}}
\newtheorem{beweis}{Beweis}
\theoremsymbol{}
\theoremstyle{change}
\theoremheaderfont{\bfseries}
\newtheorem{bemerkung}[theorem]{Bemerkung}
\newtheorem{beobachtung}[theorem]{Beobachtung}
\newtheorem{beispiel}[theorem]{Beispiel}
\newtheorem{problem}{Problem}
\theoremstyle{nonumberplain}
\renewtheorem{bemerkung*}{Bemerkung}
\renewtheorem{beispiel*}{Beispiel}
\renewtheorem{problem*}{Problem}

\newcommand{\kooperation}{}

% Algorithmen anpassen %
\renewcommand{\algorithmicrequire}{\textit{Eingabe:}}
\renewcommand{\algorithmicensure}{\textit{Ausgabe:}}
\floatname{algorithm}{Algorithmus}
\renewcommand{\listalgorithmname}{Algorithmenverzeichnis}
\renewcommand{\algorithmiccomment}[1]{\color{grau}{// #1}}

% Zeilenabstand einstellen %
\renewcommand{\baselinestretch}{1.25}
% Floating-Umgebungen anpassen %
\renewcommand{\topfraction}{0.9}
\renewcommand{\bottomfraction}{0.8}
% Abkuerzungen richtig formatieren %
\usepackage{xspace}
\newcommand{\vgl}{vgl.\@\xspace} 
\newcommand{\zB}{z.\nolinebreak[4]\hspace{0.125em}\nolinebreak[4]B.\@\xspace}
\newcommand{\bzw}{bzw.\@\xspace}
\newcommand{\dahe}{d.\nolinebreak[4]\hspace{0.125em}h.\nolinebreak[4]\@\xspace}
\newcommand{\etc}{etc.\@\xspace}
\newcommand{\evtl}{evtl.\@\xspace}
\newcommand{\ggf}{ggf.\@\xspace}
\newcommand{\bzgl}{bzgl.\@\xspace}
\newcommand{\so}{s.\nolinebreak[4]\hspace{0.125em}\nolinebreak[4]o.\@\xspace}
\newcommand{\iA}{i.\nolinebreak[4]\hspace{0.125em}\nolinebreak[4]A.\@\xspace}
\newcommand{\sa}{s.\nolinebreak[4]\hspace{0.125em}\nolinebreak[4]a.\@\xspace}
\newcommand{\su}{s.\nolinebreak[4]\hspace{0.125em}\nolinebreak[4]u.\@\xspace}
\newcommand{\ua}{u.\nolinebreak[4]\hspace{0.125em}\nolinebreak[4]a.\@\xspace}
\newcommand{\og}{o.\nolinebreak[4]\hspace{0.125em}\nolinebreak[4]g.\@\xspace}
\newcommand{\oBdA}{o.\nolinebreak[4]\hspace{0.125em}\nolinebreak[4]B.\nolinebreak[4]\hspace{0.125em}d.\nolinebreak[4]\hspace{0.125em}A.\@\xspace}
\newcommand{\OBdA}{O.\nolinebreak[4]\hspace{0.125em}\nolinebreak[4]B.\nolinebreak[4]\hspace{0.125em}d.\nolinebreak[4]\hspace{0.125em}A.\@\xspace}

% Leere Seite ohne Seitennummer, naechste Seite rechts
% Für Titelseite nötig, sonst nicht.
\newcommand{\blankpage}{
 \clearpage{\pagestyle{empty}\cleardoublepage}
}

%Booleans, die die Formatierung der Verzeichnisse im Anhang bestimmen
\newboolean{AbbildungsverzeichnisEinfuegen}
\newboolean{AlgorithmenverzeichnisEinfuegen}
\newboolean{TabellenverzeichnisEinfuegen}
\newboolean{PlatzsparendeVerzeichnisse}

% Keine einzelnen Zeilen beim Anfang eines Abschnitts (Schusterjungen)
\clubpenalty = 10000
% Keine einzelnen Zeilen am Ende eines Abschnitts (Hurenkinder)
\widowpenalty = 10000 \displaywidowpenalty = 10000
% EOF

\usepackage{moresize}

\usepackage{pifont}% http://ctan.org/pkg/pifont

\usepackage{listings}
\definecolor{deepblue}{rgb}{0,0,0.5}
\definecolor{deepred}{rgb}{0.6,0,0}
\definecolor{deepgreen}{rgb}{0,0.5,0}

%\newcommand\pythonstyle{\lstset{
\lstdefinestyle{myPython}{
	language=Python,
	basicstyle=\small,
	numbers=left,
	breaklines=true,
	tabsize=2,
	frame=lines,
	otherkeywords={self},             
	keywordstyle=\ttfamily\color{blue!90!black},
	keywords=[2]{True,False},
	keywords=[3]{ttk, row, column, sticky, fg, bg, font, lat, lon},
	keywordstyle={[2]\ttfamily\color{yellow!80!orange}},
	keywordstyle={[3]\ttfamily\color{red!80!orange}},
	emph={MyClass,__init__},          
	emphstyle=\ttfamily\color{red!80!black}, 
	stringstyle=\ttfamily\color{deepgreen},                        % Any extra options here
	showstringspaces=false            % 
	%}}
}

\colorlet{punct}{red!60!black}
\definecolor{background}{HTML}{EEEEEE}
\definecolor{delim}{RGB}{20,105,176}
\colorlet{numb}{magenta!60!black}

\lstdefinelanguage{json}{
	basicstyle=\ssmall\ttfamily,
	numbers=left,
	numberstyle=\ssmall,
	stepnumber=1,
	numbersep=8pt,
	showstringspaces=false,
	breaklines=true,
	frame=lines,
	backgroundcolor=\color{background},
	literate=
	*{0}{{{\color{numb}0}}}{1}
	{1}{{{\color{numb}1}}}{1}
	{2}{{{\color{numb}2}}}{1}
	{3}{{{\color{numb}3}}}{1}
	{4}{{{\color{numb}4}}}{1}
	{5}{{{\color{numb}5}}}{1}
	{6}{{{\color{numb}6}}}{1}
	{7}{{{\color{numb}7}}}{1}
	{8}{{{\color{numb}8}}}{1}
	{9}{{{\color{numb}9}}}{1}
	{:}{{{\color{punct}{:}}}}{1}
	{,}{{{\color{punct}{,}}}}{1}
	{\{}{{{\color{delim}{\{}}}}{1}
	{\}}{{{\color{delim}{\}}}}}{1}
	{[}{{{\color{delim}{[}}}}{1}
	{]}{{{\color{delim}{]}}}}{1},
}

