% einleitung.tex
\section{Motivation und Einleitung}
Das aktuelle Trend der Digitalisierung zeichnet sich besonders dadurch aus, dass Alltagsgegenstände smart und miteinander vernetzt werden. Folglich entlasten diese den Menschen bei alltäglichen Aufgaben. Ein Beispiel dafür ist ein Saugroboter, der vollautomatisch eine gesamte Etage saugen kann \cite{irobot2017}.\cite{kagermann2017mobilitatswende}

Im Bezug auf die Menschen hat dieser Wandel in der globalisierten Welt die Auswirkung, dass leichtere Aufgaben von den elektrischen Geräten übernommen werden, sodass dem Menschen die Verpflichtung abgenommen wird. Im Umkehrschluss müssen Menschen zunehmend komplexere Entscheidungen treffen, die mit mehr Verantwortung einhergehen. \cite{Norbisrath:62365}

Die zunehmende Globalisierung erhöht ebenfalls die Anforderungen an die zeitliche Verfügbarkeit der menschlichen Arbeitskraft bei steigender Entgrenzung von Lebens- und Arbeitswelt. Diese steigenden Anforderungen an die Menschen werden zunehmend durch technische Unterstützungssysteme kompensiert, da diese elektronische Geräte inzwischen kontextspezifische Informationen aufbereiten und übersichtlich visualisiert darstellen, sodass sie die Menschen bei ihren Aufgaben unterstützen und an Ereignisse erinnern können. Aus diesem Grund erfahren solche unterstützenden Systeme eine kontinuierlich wachsende Bedeutung.

In diesem Zusammenhang haben sich SmartMirrors als hilfreiche Geräte zur Datenvisualisierung und Unterstützung des Menschen herausgestellt. So liefern sie jederzeit die aktuellsten Nachrichten, damit der Besitzer immer auf dem aktuellen Stand des Weltgeschehens ist. Im besonderen können sie Demenz-kranken helfen, indem die wichtigsten Dinge, die ein Patient nicht vergessen darf, jederzeit über den SmartMirror abrufbar sind. Allerdings ist es wie häufig bei neuen oder medizinisch relevanten Produkten so, dass diese derzeit in einem hohen Preissegment liegen \cite{marketresearch16smartmirror}.

Aus diesem Grund ist es das Ziel dieser Ausarbeitung, zu zeigen, dass solch ein SmartMirror auch mit einem kleineren Budget und den Kenntnissen eines Informatikstudenten realisierbar ist.

Grundsätzlich ist die Ausarbeitung so aufgebaut, dass zunächst ein Überblick über die im Spiegel integrierten Hardwarekomponenten gegeben wird. Anschließend beschreibt die Ausarbeitung den Aufbau der Softwarekonstruktion und das Zusammenspiel zwischen Hardware und Software. Ein Resümee über den Projektverlauf und ein Fazit bilden den Abschluss der Arbeit.



