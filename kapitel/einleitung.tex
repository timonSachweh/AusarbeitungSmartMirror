% einleitung.tex
\chapter{Motivation und Einleitung}
"Vergisst du auch ständig den Geburtstag deiner Großeltern?", ist der Leitsatz unseres Projekts \textit{SmartMirror}. \\

Mir persönlich passiert es immer wieder, dass ich im Laufe des Tages feststellen muss das ich einen Geburtstag oder einen anderen wichtigen Termin vergessen habe. \\
Und damit dies nicht immer wieder passiert, haben wir den \textit{SmartMirror} entwickelt. \\

Wir haben uns schnell für dieses Thema entschieden, da es so ziemlich jede Person betrifft und nach Fertigstellung eine große Bereicherung für den Tagesablauf bietet. \\
Das Projekt bündelt alle wichtigen Informationen zu einer Tagesuhrzeit, nämlich direkt nachdem aufstehen und leitet den Benutzer in einen geregelten Ablauf über. \\

Wie der Name \textit{SmartMirror} schon andeutet, handelt es sich bei unserem Projekt um einen extern-physischen intelligenten Spiegel in einem geschlossenen System, der Funktionen wie das Anzeigen verschiedener Temperaturen oder News-Feed's, einen Terminkalender oder Geburtstagskalender bereitstellt. \\
Die Funktionen werden in den folgenden Kapitel ausführlich erläutert. \\
Der \textit{SmartMiror} bietet verschiedene Upgrade-Möglichkeiten und ist somit für zukünftige Projekte offen. \\

Die Ausarbeitung zu diesem Proseminar-Thema führt im 2. Kapitel zunächst durch die Fertigung des \textit{SmartMirror's}. Hierfür erklären wir wichtige Hardwarekomponenten, technische Eigenschaften und natürlich die Herstellung des Spiegels. \\

Im 3. Kapitel gehen wir dann näher auf das Herzstück des \textit{SmartMirror's} ein. Wir haben passend zu unserem Projekt die Software geschrieben. Dieses Kapitel veranschaulicht das Konzept und die einzelnen Funktionen, welche mit der Hardware des Spiegels interagieren. \\
Den Abschluss bildet ein Demovideo des Spiegels.



