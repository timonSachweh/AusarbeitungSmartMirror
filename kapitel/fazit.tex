\section{Fazit und Ausblick}

Zusammenfassend lässt sich anführen, dass am Ende der Ausarbeitung ein fertiger SmartMirror mit einem Budget von unter \EUR{300} realisiert wurde. 
Wie schon in der Einleitung erwähnt, liegen käufliche SmartMirrors tendenziell eher in einem Preissegment von über \EUR{1000}, sodass das angestrebte Ziel der Arbeit, bei dem selbstgebauten Spiegel die Kosten um ein Vielfaches zu senken, erreicht wurde.

Dieses Ziel wurde dadurch unterstützt, dass ausschließlich Kosten für die Hardwarekomponenten anfielen, die dann selbstständig zusammengebaut wurden. Bezüglich der Software fielen gar keine Kosten an, da diese komplett im Rahmen der Seminararbeit entwickelt wurde.
Diese Aspekte haben den Nebeneffekt, dass die Form des Spiegels, sowie die Software exakt für die Ausarbeitung angepasst werden konnte und damit die Komponenten auch optimal aufeinander abgestimmt werden konnten.

Neben diesem Ergebnis hat die Erarbeitung des Spiegels zu einer Menge an Erfahrungen geführt, die vor allem mit dem Fortschritt innerhalb der Erarbeitungsphase einhergegangen sind.
So ist eine Planung der Teilabschnitte des Projektes notwendig, damit auch festgesetzte Deadlines existieren, auf die sich die anderen Leute im Team verlassen können, dass die Komponenten bis zu dem Datum wirklich fertig sind. Dadurch wird die Effizienz aller Personen im Team gesteigert und man kann Teile des in \autoref{sec:imEinsatz} beschriebenen Ablaufes schon mitten im Projektverlauf durchführen, um zu Testen, ob bis dahin der Projektfortschritt eingehalten wurde. 
Die Planung hat außerdem den Vorteil, dass sich  Gedanken darüber gemacht werden, wie reagiert wird, wenn das Projekt nicht den vollen geplanten Umfang in der Zeit erreichen wird. In diesem Fall wurde die Ausarbeitung so strukturiert, dass als erstes der Spiegel mit den nötigsten Funktionen ausgestattet wurde und es im Anschluss daran verschiedene Punkte gab, um die der \textit{SmartMirror} erweitert werden konnte. Als Beispiel gab es den erweiterten Punkt, dass eine Web-Oberfläche zur Konfigurierung der Anwendung realisiert werden sollte, die aber bis zum Zeitpunkt der Abgabe nicht fertig gestellt wurde. Dagegen wurde aber der optionale Bewegungssensor noch in den Spiegel integriert und softwaretechnisch angebunden, sodass zumindest ein Teil der erweiterten Planung in den \textit{SmartMirror} integriert werden konnte.

Neben diesen positiven Erfahrungen sind aber auch negative Aspekte in den Vordergrund geraten, die bei weiteren Projekten schon von Anfang an mit einbezogen werden sollten. Beispielsweise wurde in diesem Projekt nicht sofort die gesamte Hardware beschafft und auf technische Lauffähigkeit überprüft. Das hatte zur Folge, dass zu Beginn mit der Software gestartet werden musste und diese für einen fiktiven Monitor konstruiert wurde, sodass die UI nach dem Zusammenbau der Hardware noch einmal umgebaut werden musste. Außerdem lag für die Ausarbeitung zuerst nur ein defekter Monitor bereit, was jedoch erst nach Einbau der ganzen Komponenten aufgefallen ist, da vorher kein Funktionscheck durchgeführt wurde.

Aus all diesen Punkten lässt sich somit zusammenfassend sagen, dass das zu Grunde gelegte Ziel des geringeren Preises mit großem Abstand erfüllt wurde und zusätzliche Erfahrungen im Rahmen der Projektplanung und Projektdurchführung gesammelt wurden.

Ausgehend von dieser Basis wird das Projekt zudem in weiteren Schritten noch außerhalb der Ausarbeitung fortgeführt werden, sodass die Web-Oberfläche fertig gestellt wird, eine Gestenerkennung zur Steuerung der Oberfläche integriert und eine Kamera zur Personenidentifizierung angebracht werden.